\documentclass{article}

% Language setting
% Replace `english' with e.g. `spanish' to change the document language
\usepackage[english]{babel}

% Set page size and margins
% Replace `letterpaper' with `a4paper' for UK/EU standard size
\usepackage[letterpaper,top=2cm,bottom=2cm,left=3cm,right=3cm,marginparwidth=1.75cm]{geometry}

% Useful packages
\usepackage{amsmath}
\usepackage{graphicx}
\usepackage[colorlinks=true, allcolors=blue]{hyperref}
\usepackage{amsfonts}

\title{Linear Algebra}
\date{}

\begin{document}
\maketitle

\tableofcontents

\newpage
\section{Matrix}
Matrix $A \in \mathbb{R}^{m \times n}$ defines a linear transformation from $\mathbb{R}^n$ to $\mathbb{R}^m$.\\
Matrix $A$ is composed of column vectors $\mathbf{a}_j$, each of which is the image of the $j$-th standard basis vector after applying the transformation $A$.
\[
    A
\]
Expanded form~\ref{eq:matrix}


\newpage
\section{Matrix Operation}


\subsection{Inner Product}
The inner product (dot product) of two vectors $\mathbf{a}, \mathbf{b} \in \mathbb{R}^n$ is a scalar:
\[
    \mathbf{a}^T \mathbf{b} = \sum_{i=1}^n a_i b_i
\]
Expanded form~\ref{eq:innerprod}.

\subsection{Outer Product}
The outer product of two vectors $\mathbf{a} \in \mathbb{R}^m$ and $\mathbf{b} \in \mathbb{R}^n$ is an $m \times n$ matrix:
\[
    \mathbf{a} \mathbf{b}^T
\]
Expanded form~\ref{eq:outerprod}.

\subsection{Matrix-Vector Multiplication}
Apply linear transformation $A \in \mathbb{R}^{m \times n}$ to vector $\mathbf{x} \in \mathbb{R}^n$ and produce a new vector $A\mathbf{x} \in \mathbb{R}^m$
\[
    A\mathbf{x} = \sum_{j=1}^n x_j \mathbf{a}_j
\]{
Expanded} form~\ref{eq:matvec}.

\subsection{Matrix-Matrix Multiplication}
Apply transformation $B$, then apply transformation $A$.
\[
    (AB)_{*j} = A\mathbf{b}_j
\]
Expanded form~\ref{eq:mul}.

\subsection{Associative Law}
\[
    (AB)C = A(BC)
\]

\subsection{Transpose}
\[
    A^T = (a_{ji})
\]
\textbf{Transpose Distribution Rule:}
\[
    (AB)^T = B^T A^T
\]
Expanded form~\ref{eq:trans}.

\subsection{Inverse}
Reverses the transformation of $A$
\[
    A A^{-1} = A^{-1} A = I_n
\]
\textbf{Inverse of a Product Rule:}
\[
    (AB)^{-1} = B^{-1}A^{-1}
\]
Calculation~\ref{subsec:gauss}.


\newpage
\section{Special Matrix}

\subsection{Identity Matrix}
The diagonal entries are all 1, and all off-diagonal entries are 0.\\
\textbf{Geometric Interpretation:} No transformation
\[
    (I_n)_{ij} =
    \begin{cases}
        1, & i = j \\
        0, & i \neq j
    \end{cases}
\]
Expanded form~\ref{eq:iden}.

\subsection{Diagonal Matrix}
The diagonal entries can be any value, and all off-diagonal entries are 0.\\
\textbf{Geometric Interpretation:} Scale the $i$-th standard basis vector by $d_i$
\[
    (D_n)_{ij} =
    \begin{cases}
        d_i, & i = j \\
        0, & i \neq j
    \end{cases}
\]
Expanded form~\ref{eq:diagm}.\\
\textbf{Inverse of a Diagnal Matrix:}
\[
    (D_n^{-1})_{ij} =
    \begin{cases}
        \frac{1}{d_i}, & i = j \\
        0, & i \neq j
    \end{cases}
\]
Expanded form~\ref{eq:diagminv}.

\subsection{Orthogonal Matrix}
All column vectors are unit vectors and orthogonal to each other.\\
\textbf{Geometric Interpretation:} Preserves lengths and angles
\[
    Q^T Q = Q Q^T = I_n
\]
\[
    Q^T = Q^{-1}
\]

\subsection{Positive Definite Matrix}
A symmetric matrix $A \in \mathbb{R}^{n \times n}$ is called \textbf{positive definite} if for any nonzero vector $\mathbf{x} \in \mathbb{R}^n$,
\[
    \mathbf{x}^T A \mathbf{x} > 0
\]
\textbf{Properties:}
\begin{itemize}
    \item All eigenvalues of $A$ are positive.
    \item $A$ is invertible, and $A^{-1}$ is also positive definite.
\end{itemize}

\subsection{Positive Semi-Definite Matrix}
A symmetric matrix $A \in \mathbb{R}^{n \times n}$ is called \textbf{positive semi-definite} if for any vector $\mathbf{x} \in \mathbb{R}^n$,
\[
    \mathbf{x}^T A \mathbf{x} \geq 0
\]
\textbf{Properties:}
\begin{itemize}
    \item All eigenvalues of $A$ are non-negative.
\end{itemize}


\newpage
\section{Matrix Decomposision}

\subsection{Eigenvector \& Eigenvalue}

A unit \textbf{eigenvector} $\|\mathbf{v}\| = 1$ of a matrix $A \in \mathbb{R}^{n \times n}$ is scaled by its corresponding \textbf{eigenvalue} $\lambda$ when the transformation $A$ is applied.
\[
    A\mathbf{v} = \lambda \mathbf{v}
\]
Calculation:
\[
    \det(A - \lambda I) = 0
\]

\subsection{Spectral Decomposition}

Symmetric matrix $A \in \mathbb{R}^{n \times n}$ can be decomposed as:
\[
    A = Q \Lambda Q^T
\]
where:
\begin{itemize}
    \item $Q$ is an orthogonal matrix whose columns are the eigenvectors of $A$.
    \item $\Lambda$ is a diagonal matrix whose diagonal entries are the eigenvalues of $A$.
\end{itemize}
\textbf{Geometric Interpretation:}
\begin{enumerate}
    \item $Q^T$: Change the basis of the vector space to the eigenvector basis.
    \item $\Lambda$: Apply a scaling transformation along the new axes by the eigenvalues.
    \item $Q$: Change the basis of the vector space back to the standard basis.
\end{enumerate}

\subsection{Eigen Decomposition}

Square matrix $A \in \mathbb{R}^{n \times n}$ can be decomposed as ($A$ is \textbf{similar} to $D$):
\[
    A = P \LambdaP ^{-1}
\]
where:
\begin{itemize}
    \item $P$ is a matrix whose columns are the eigenvectors of $A$.
    \item $\Lambda$ is a diagonal matrix whose diagonal entries are the eigenvalues of $A$.
\end{itemize}
\textbf{Geometric Interpretation:}
\begin{enumerate}
    \item $P^{-1}$: Change the basis of the vector space to the eigenvector basis.
    \item $\Lambda$: Apply a scaling transformation along the new axes by the eigenvalues.
    \item $P$: Change the basis of the vector space back to the standard basis.
\end{enumerate}

\subsection{Singular Value Decomposition}

\subsubsection{Singular Vector \& Singular Value}
For any matrix $A \in \mathbb{R}^{m \times n}$, the \textbf{singular values} (the square roots of the eigenvalues of $A^T A$ or $AA^T$) measure how much $A$ stretches unit vectors in different directions, with the corresponding \textbf{right singular vectors} (eigenvectors of $A^T A$) indicating the directions of maximum stretch in the input space, and the \textbf{left singular vectors} (eigenvectors of $AA^T$) indicating the directions of maximum stretch in the output space.

\subsubsection{SVD Formula}
Any matrix $A \in \mathbb{R}^{m \times n}$ can be decomposed as:
\[
    A = U \Sigma V^T
\]
where:
\begin{itemize}
    \item $V \in \mathbb{R}^{n \times n}$ is an orthogonal matrix whose columns are the \textbf{right singular vectors} of $A$.
    \item $U \in \mathbb{R}^{m \times m}$ is an orthogonal matrix whose columns are the \textbf{left singular vectors} of $A$.
    \item $\Sigma \in \mathbb{R}^{m \times n}$ is a diagonal matrix whose diagonal entries are the \textbf{singular values} of $A$.
\end{itemize}
\textbf{Geometric Interpretation:}
\begin{enumerate}
    \item $V^T$: Change the basis of the input space from the standard basis to the right singular vector basis.
    \item $\Sigma$: Apply scaling transformation by the singular values and map from the right singular vector basis to the left singular vector basis.
    \item $U$: Change the basis of the output space from the left singular vector basis to the standard basis.
\end{enumerate}


\subsubsection{SVD Derivation}
Assume that matrix $A \in \mathbb{R}^{m \times n}$ can be decomposed as:
\[
    A = U \Sigma V^T
\]
where $U \in \mathbb{R}^{m \times m}$ and $V \in \mathbb{R}^{n \times n}$ are orthogonal matrices, and $\Sigma \in \mathbb{R}^{m \times n}$ is a diagonal matrix.

\paragraph{Compute $A^T A$ and $AA^T$:}
\[
    A^T A = (U \Sigma V^T)^T (U \Sigma V^T) = V \Sigma^T U^T U \Sigma V^T = V \Sigma^T \Sigma V^T
\]
\[
    AA^T = (U \Sigma V^T)(U \Sigma V^T)^T = U \Sigma V^T V \Sigma^T U^T = U \Sigma \Sigma^T U^T
\]
Notice that these expressions are in the form of eigen decomposition:
\begin{itemize}
    \item $A^T A = V (\Sigma^T \Sigma) V^T$ where $V$ is orthogonal and $\Sigma^T \Sigma$ is diagonal
    \item $AA^T = U (\Sigma \Sigma^T) U^T$ where $U$ is orthogonal and $\Sigma \Sigma^T$ is diagonal
\end{itemize}
Therefore: $V$ contains the eigenvectors of $A^T A$, and $\sigma_i = \sqrt{\lambda_i}$. $U$ contains the eigenvectors of $AA^T$, and $\sigma_i = \sqrt{\lambda_i}$.


\newpage
\section{Quadratic Form}

\subsection{Quadratic Form}
For a vector $\mathbf{x} \in \mathbb{R}^n$, a quadratic form is:
\[
    q(\mathbf{x}) = \mathbf{x}^T A \mathbf{x}
\]
where $A \in \mathbb{R}^{n \times n}$ is a symmetric matrix.

\subsection{Geometric Interpretation}
The quadratic form $q(\mathbf{x}) = \mathbf{x}^T A \mathbf{x}$ describes the geometry of a quadric surface.

\paragraph{Apply Spectral Decomposition:}
\[
    q(\mathbf{x}) = \mathbf{x}^T (Q \Lambda Q^T) \mathbf{x} = (\mathbf{x}^T Q) \Lambda (Q^T \mathbf{x})
\]
Define $\mathbf{y} = Q^T \mathbf{x}$:
\[
    q(\mathbf{y}) = \mathbf{y}^T \Lambda \mathbf{y} = \sum_{i=1}^n \lambda_i y_i^2
\]
\begin{enumerate}
    \item The \textbf{eigenvectors} of $A$ (columns of $Q$) define the \textbf{principal axes} of the geometric shape.
    \item The \textbf{eigenvalues} $\lambda_i$ of $A$ \textbf{scales} the length of each semi-axis by $1/\sqrt{\lambda_i}$.
\end{enumerate}


\newpage
\section{Calculation}

\subsection{Gaussian Elimination}
\label{subsec:gauss}

Gaussian elimination is an algorithm for solving systems of linear equations, finding the rank, and calculating the inverse of a matrix. The process consists of two main steps:

\begin{enumerate}
    \item \textbf{Forward Elimination:} Transform the matrix into an upper triangular form using \textbf{elementary row operations}:
    \begin{itemize}
        \item Swap two rows.
        \item Multiply a row by a nonzero scalar.
        \item Add a multiple of one row to another row.
    \end{itemize}
    \item \textbf{Back Substitution:} Solve for the variables starting from the last row and moving upwards.
\end{enumerate}


\newpage
\appendix
\section{Expanded Form}

\begin{itemize}

    \item \textbf{Matrix:}
    \begin{equation}
        A =
        \left[
            \begin{array}{ccc}
                | &        & | \\
                \mathbf{a}_1 & \cdots & \mathbf{a}_n \\
                | &        & |
            \end{array}
        \right]
        \label{eq:matrix}
    \end{equation}

    \item \textbf{Inner Product:}
    \begin{equation}
        \mathbf{a}^T \mathbf{b} =
        \left[
            \begin{array}{ccc}
                a_1 & \cdots & a_n
            \end{array}
        \right]
        \left[
            \begin{array}{c}
                b_1 \\
                \vdots \\
                b_n
            \end{array}
        \right]
        = a_1 b_1 + \cdots + a_n b_n
        \label{eq:innerprod}
    \end{equation}

    \item \textbf{Outer Product:}
    \begin{equation}
        \mathbf{a} \mathbf{b}^T =
        \left[
            \begin{array}{c}
                a_1 \\
                \vdots \\
                a_m
            \end{array}
        \right]
        \left[
            \begin{array}{ccc}
                b_1 & \cdots & b_n
            \end{array}
        \right]
        =
        \left[
            \begin{array}{ccc}
                a_1 b_1 & \cdots & a_1 b_n \\
                \vdots & \ddots & \vdots \\
                a_m b_1 & \cdots & a_m b_n
            \end{array}
        \right]
        \label{eq:outerprod}
    \end{equation}

    \item \textbf{Matrix-Vector Multiplication:}
    \begin{equation}
        A\mathbf{x} =
        \left[
            \begin{array}{ccc}
                | &        & | \\
                \mathbf{a}_1 & \cdots & \mathbf{a}_n \\
                | &        & |
            \end{array}
        \right]
        \left[
            \begin{array}{c}
                x_1 \\
                \vdots \\
                x_n
            \end{array}
        \right]
        =
        x_1\mathbf{a}_1 + x_2\mathbf{a}_2 + \cdots + x_n\mathbf{a}_n
        \label{eq:matvec}
    \end{equation}

    \item \textbf{Matrix-Matrix Multiplication:}
    \begin{equation}
        AB =
        \left[
            \begin{array}{ccc}
                | &        & | \\
                A\mathbf{b}_1 & \cdots & A\mathbf{b}_p \\
                | &        & |
            \end{array}
        \right]
        \label{eq:mul}
    \end{equation}

    \item \textbf{Transpose:}
    \begin{equation}
        \left[
            \begin{array}{ccc}
                a_{11} & \cdots & a_{1n} \\
                \vdots & \ddots & \vdots \\
                a_{m1} & \cdots & a_{mn}
            \end{array}
        \right]^T
        =
        \left[
            \begin{array}{ccc}
                a_{11} & \cdots & a_{m1} \\
                \vdots & \ddots & \vdots \\
                a_{1n} & \cdots & a_{mn}
            \end{array}
        \right]
        \label{eq:trans}
    \end{equation}

    \item \textbf{Identity Matrix:}
    \begin{equation}
        I_n =
        \left[
            \begin{array}{cccc}
                1 & 0 & \cdots & 0 \\
                0 & 1 & \cdots & 0 \\
                \vdots & \vdots & \ddots & \vdots \\
                0 & 0 & \cdots & 1
            \end{array}
        \right]
        \label{eq:iden}
    \end{equation}

    \item \textbf{Diagonal Matrix:}
    \begin{equation}
        D_n =
        \left[
            \begin{array}{cccc}
                d_1 & 0 & \cdots & 0 \\
                0 & d_2 & \cdots & 0 \\
                \vdots & \vdots & \ddots & \vdots \\
                0 & 0 & \cdots & d_n
            \end{array}
        \right]
        \label{eq:diagm}
    \end{equation}

    \item \textbf{Inverse of Diagonal Matrix:}
    \begin{equation}
        D_n^{-1} =
        \left[
            \begin{array}{cccc}
                \frac{1}{d_1} & 0 & \cdots & 0 \\
                0 & \frac{1}{d_2} & \cdots & 0 \\
                \vdots & \vdots & \ddots & \vdots \\
                0 & 0 & \cdots & \frac{1}{d_n}
            \end{array}
        \right]
        \label{eq:diagminv}
    \end{equation}

\end{itemize}

\end{document}